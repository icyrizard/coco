    \documentclass[a4paper]{article}
\usepackage[a4paper]{geometry}
\usepackage[utf8]{inputenc}
\usepackage{listings}
\usepackage{graphicx}
\usepackage[all]{xy}
\usepackage{amsmath}
\usepackage{makeidx}
\marginparsep = 10pt
\author{Rik van der Kooij - 6064965\\
Richard Torenvliet - 6138861}
\title{Compiler Construction - Creating the CiViC Compiler}
\begin{document}
\lstset{
    numbers=left,
    tabsize=2,
}


\maketitle
\tableofcontents

\abstract{Civilized C Compiler}

\section{Introduction}
This document describes yet another implementation of the CiviC compiler for
the CiviC-VM. The creation of the CiviC compiler consists of different phases
in order to parse program text to assembly which in turn the CiviC-VM can
interpret. The CiviC language has a C-like structure. By implementing the
compiler, all the steps that involve the creation of a compiler can be learned.
Because CiviC takes use of the CiviC-VM, we do not have to worry about
different architectures the program runs on. To aid in the creation of the
compiler, we take use of the framework that comes with the assignment. The framework provides
functionalities to create phases, to construct an AST and comes with helper
functions.

A starting point is defining the AST(Abstract Syntax Tree), which describes the
hierarchy of ``Nodes". The formal definition of the AST is stated in Backus
Nauer form in the CiviC Language Manuel. The next step is creating the Lex and
Yacc to parse a file and assemble the AST. This AST is used in the next phases
like ``desugaring", ``type checking" and ``code generation". These phases are
explained in the following sections.

\section{List and HashMap}
A list data structure with an dynamic size would help out alot. An example
would be in  desuggaring of for loops where all new vardecs could then be
stored in that list. We created our own linked list implementation in the file:
`framework/src/framework/list\_hash.c'. All function names should be enough to
indicate what they are doing.

Another usefull data structure would be an hash map. We also created a simple
hash map implementation in the same file as the linked list. The hash map can
be used to create simple rename rules for variables. For example for loop
desugaring requires to rename all loop variables. In the hash map we can use
the key for the old value for lookup and the value for the new value.

\section{Abstract Syntax Tree}
An abstract syntaxtree is a representation of the structure of components that
are involved in the definition of a language.
The abstract syntaxtree is represented in XML form. In the XML
file(\texttt{src/global/ast.xml}) the hierarchy and relation between nodes can
be expressed. To implement the CiviC language, the CiviC reference manual has
to be carefully followed. The language is defined in Backus Nauer form which is
turned in to XML. The XML follows a specific pattern in order to describe the tree.
For example, an assign node(see reference document) is described by the
following.

\begin{lstlisting}[language=XML]
 <node name="Assign">
  <sons>
    <son name="Let">
      <targets>
        <target mandatory="yes">
          <node name="Var" />
          <phases>
            <all />
          </phases>
        </target>
      </targets>
    </son>
    <son name="Expr">
      <targets>
        <target mandatory="yes">
          <set name="Expr" />
          <phases>
            <all />
          </phases>
        </target>
      </targets>
    </son>
  </sons>
  <attributes />
</node>
\end{lstlisting}

An important thing to note is that <sons> represent other nodes in the ast.xml.
By declaring other nodes a relation between these nodes is created. To create a cyclic
relation(needed for lists, e.g statement lists), the node that is referred to can refer back to the parent node.

Another field can be used if a node is a ctype, those are described in the \texttt{<attributes>} field.\\
\texttt{<type name="String" ctype="char*" init="NULL" copy="function" />}
\\
The AST can be visually represented in a Directed Cyclic Graph, the resulting
tree can be seen in figure \ref{fig:ast.png}.

\begin{figure}[h!]
    \includegraphics[width=15cm]{../framework/doc/ast.png}
\label{fig:ast.png}
\caption{The resulting AST}
\end{figure}

The framework generates functions to create these nodes which can be used in
the Yaccer matches sequence of tokens to nodes. To construct the AST in memory.
Which then can be used to perform semantically check the code. The functions
that are generated also follow a specific syntax:\\
\texttt{TBmake<NODENAME>\_<SON\_NAME/ATTRIBUTE\_NAME>}

\section{Lex and Yacc}
This phase forms the base of the compiler. The first step is to do a lexical
analysis with Lex, it matches a sequence of characters that form will form a
token, also non allowed characters are detected. A tokenstream is generated
form the code and are provided to the Yaccer, a sequence of these map to a
certain node(exactly the nodes we described in the ast.xml and the CiviC
reference). If a successful match is found, the according \texttt{TBmake..(\$1, \$2, \$3)} function is called. The
matched tokens are used as parameters. Then end result will be the complete AST
in memory.

\section{Phases and Traversing}
The framework provides ways to add compiler phases to the process. First,
adding the phase to \texttt{src/global/myphases.mac}, then adding a new folder
in \texttt{src/} which corresponds to the name of the phase. In
\texttt{Makefile.Targets} the name of the object file needs to added, the
framework will then compile the new phase. The phases are executed one after
the other in the order of which they are placed in \texttt{src/global/phase.mac}.

Traversing the tree is an important aspect to target nodes, e.g rewriting for
loops in to while loops will only target the while nodes. Which nodes are
targeted in a phase is defined in the \texttt{ast.xml}, also nodes that we
should not traverse can be expressed.

The framework performs functioncalls which are also constrained to a specific
syntax: \texttt{<PHASENAME><nodename>(arg *node, info *node);}. The arg is the
node type that we want to traverse. In a lot of cases the program needs to
alter or traverse a node inside the given node. For that the framework provides
macros to extract this data. Which follow the following syntax:
\texttt{<NODENAME>\_<NODEFIELD>(*node)}.

The info node object is used to communicate between these functions. It can be
used to contain anything to create a state of the current traversal phase. The
following sections explain the usage of this object if needed.

\section{Desugaring}
Due to the fact that declarations and initializations of variables are not
always possible, these need to be rewritten to a separate declaration or separate
initialization.

\subsection{Global variables to \_\_init}
The first desugaring step is to rewrite the initialization of global variables
to a new \_\_init function, this is in fact "code generation". The reason for
this is that ultimately the assembly instructions need to be placed under a
label/function. In addition, the global variables need to be initialized before any
other instruction is invoked, simply because of the fact that we expect them to
hold the right variable before anything else.

This phase does imply that the \_\_init function is called first by the
CiviC-VM and secondly that \_\_init can not be declared by the user.
The node of interest is the \texttt{GlobalDef}, which is defined in the CiviC
manual by:\\
\textit{GlobalDef $\Rightarrow$ [ \texttt{export} ] Type Id [ = Expr ]}

In the event of an Expr, which means that an assignment is present. A new
assignment expression is created and prepended to a list in the info object.
The list of new assign nodes is added in the end to a new function called
\_\_init. See psuedo code in figure \ref{fig:init}.

\begin{figure}[h]
    \begin{lstlisting}[language=C]
    info = {
        list *globaldefs;
    };

    INITglobaldef (arg, info){
        /* return if no assignment*/
        if arg->expr:
            return

        /* store a new expression node */
        info->globaldefs.prepend(new_expr_node);

        /* remove the assignment */
        arg->expr = NULL;
    }

    /* add init to top of statements */
    add_init(info);
    \end{lstlisting}
    \caption{Globaldefs to \_\_init}
    \label{fig:init}
\end{figure}

\subsection{Forloop variable initialization desuggering}
All forloops start with a declaration part. This part will be split from the
for loop much like we did with the global variables. The declaration of the
loop variables is moved to the top of the function. The loop variables can be
seen as though they are defined in a different scope. This means that we had to
rename it to allow multiple local variables of the same name.

\begin{lstlisting}[language=C]
for(int i = 0, 10, 1){
    i = i + 2;
    for(int j = 0, 10, 1){
        i = i + 1;
    }
}
\end{lstlisting}
\begin{lstlisting}[language=C]
int $i1;
int $j2;
for($i1 = 0, 10, 1){
    $i1 = $i1 + 2;
    for($j2 = 0, 10, 1){
        $i1 = $i1 + 1;
    }
}
\end{lstlisting}

The list described in section %ref
is used to construct a list of variable declarations with the renamed
variables, i.e. \texttt{\$<VARNAME><NESTLEVEL>}. For all occurrences of variables with
the old name have to be renamed to the new one. The solution is to create a
hashmap: \texttt{"old\_varname":"new\_varname"}. If a variable of
\texttt{"old\_varname"} is encountered it has to be renamed to the new varname.

At the end of the traversal, the new vardeclist is added to the current
vardeclist of the current function.

\section{Semantic Analysis}
Semantic analysis is where we add semantic information and check whether or not
the given program is valid. After this phase the program should be valid and no
further checking should be needed.

\subsection{Context Analysis}
Before we can do any type checking we should have a convenient way to get the
type of an identifier. This is what we are going to do by linking each
identifier with its declaration.

Declarations of variables should always occur before use. This makes linking
them fairly easy. Each time we find a new declaration we store it either in a list
for global or local variables. When a variable is used we simply get the
declaration from the list. Duplicate declarations or missing declarations will
be reported.

Functions on the other hand can be used before their declaration. For this we
simply kept track of all function calls where we didn't see a declaration for
yet. Op on reaching a declaration we would link all stored function calls. Error
messages of non declared function calls is done at the end.

\subsection{Type Checking}
Type checking makes sure that all functions and operations are applied to the
correct types. The implementation rests on the last phase which enables to get
the type of the identifiers. This way we can throw the type upwards when we traverse the tree. In every
traversal of a node we can simply traverse the children and get their types.
When we have the types of the children it is merely checking whether or not the
types are ok.

An example for type checking for the assign node
is layed out in figure \ref{fig:type}. The type of the left-side of the
assignment can easily be checked, the right-side is of type Expr. The type of
every Expr is tracked by placing it in the info node. The left and right side
can ultimately be compared in the Assign Node.

\begin{figure}[h]
\xymatrix{
            & *+++o[F]\txt{ASSIGN \\ type-left == type-right ?} \ar[dl] \ar[dr]&  \\
*+++o[F]{VAR, TYPE\_int} & & *+++o[F]{EXPR} \\
            & & ... \ar[d] \ar@/_/[u]_{\txt<7pc>{TYPE upwards}} \\
            & & ...  \ar@/^/[u]_{\txt<9pc>{TYPE upwards}} \\
}
\label{fig:type}
\caption{Assignment node type checking.}
\end{figure}

\subsection{Loop Rewriting}
We made a little phase to rewrite do while and for loops to while loops. As we
thought it might make the rest of the code easier. Do while loops simply have
their body copied above the new while loop and for loops are have their
initialization above the while and the increment at the end of the loop.
The reason for rewriting the loops to for loop, is that the assembly of all
variations of loops will be the same. That makes it easier in the assembly
phase.

This phase will involve the deletion/replacement of nodes by new nodes. See
fig. \ref{fig:fortowhile} and fig. \ref{fig:dotowhile} for an example of how a
for loop to while loop looks like.

\begin{figure}[h!]
\centering
\begin{lstlisting}[language=C]
int i;
for(i = 0; 5, 1){

}
\end{lstlisting}
\begin{lstlisting}[language=C]
int i = 0;
while(i < 5){

    i = i + 1;
}
\end{lstlisting}
\caption{For to while loop}
\label{fig:fortowhile}
\end{figure}

\begin{figure}[h!]
\centering
\begin{lstlisting}[language=C]
do {

} while(i);
\end{lstlisting}

\begin{lstlisting}[language=C]
while(i){

}
\end{lstlisting}
\caption{Do while to while loop}
\label{fig:dotowhile}
\end{figure}

\subsection{Conjunction and Disjunction replacement}
In this phase we rewrite the logic conjunction and disjunction to branches. One
of the rewrites looked as follows:
\begin{lstlisting}[language=C]
/* from this */
var = expr && expr

/* to this */
temp = expr
if(temp)
    temp = expr
var = temp
\end{lstlisting}

If we take line 8 to be our old statement we only have to change the expression
of the assignment and create new statements above it. This is a problem since
the statement list can only been traversed forwards. Once arrived at a
statement you can't go back. Which in turn means you only append statements
after the current one.

We tackled this by not picking line 8, but line 5 as our old statement. Replacing
both the let and expression of the assignment. We can then easily insert the
other three statements to the statement list.

\section{Code Generation}
In the code the last three phases (code generation, peephole optimization and
assembly printing) are put in one file. This was unfortunately due to time
constraints. It made our lives a little easier as we did not have to worry
about passing the assembly instructions to a new phase.

%TODO: we gebruiken voor elke assembly onderdeel een lijst (instr, import, ..)
The civic assembly can be described as a list of instructions followed by
a global variable, a constant, an import and an export table. In our code each
of these will be represented as a list which are filled when traversing the
syntax tree. All these lists are initialized in the info object.

For the creation of Assembly instructions we need a new type of Node, namely an
AssemblyInstr Node. These can be printed at the and of the traversal.
Asembly instructions have arguments which has an undefined argument length.
Because of this, an AssemblyInstr node has an ArgumentList. With these type of
nodes, all the instructions can be created.

\subsection{.export table}
\texttt{.export "<func\_name>" <return Type> [<paramtypes>, ..] <func\_name>} \\
The first step is to create a list of imports, since the main and \_\_init function
needs to be visible to the VM as a starting point, also other functions that are
declared with export also need to be added to this list. The list can be build
by purely traversing every \texttt{FunDef} nodes, in case of an export, we
create and add a new AssemblyInstruction to the list of imports.

\subsection{.const table}
\texttt{.const <BasicType> <value>} \\
The const table is used for constant variables in an Expr node. All operations
are performed on registers, these constant variables do not have a register
assigned yet. Therefore contantpool is created. This means traversing,
\texttt{Var}, \texttt{Float}, \texttt{Int}, \texttt{Bool} nodes. Hereby keeping
in mind that we can optimize with direct constant loading. In this case, we
don't need to create a new assembly instruction in the constant table. Thus
sparing bytes.

\subsection{.global table}
\texttt{.global Type}\\
All the \texttt{GlobalDec} nodes are traversed and new nodes are added to the import
list. The value is not important, just the \#i-th place of the variable. In case
of a load of the variable this index is the offset that the VM uses to target
the right variable register. The reason why an imports table is created is that
they are accessible from any place in the code.

\subsection{esr}
\texttt{esr L} \\
In the encounter of a \texttt{FunDef} node a new function label is created, and
added to the localvars list. The localvars list keeps track of all the vardecs
in a function. The \texttt{L} includes the amount of parameters and the list of
vardecs. The VM reserves space for these parameters. This is easily done by
counting the amount of parameters(which is a list), and counting the amount of
vardecs (which is also a list). Those two together form the nummer L.

\subsection{Loads}
\texttt{Load}


%TODO: instructies maken we zo!
\section{Peephole Optimization}
Our peepholer looks at two instructions at the time. We optimize a consecutive
load and store to the same register by removing both instructions.
Another optimization is done by changing 'load 0' to 'load\_0'

%TODO als je nog iets weet


\section{Conclusion}
%TODO Awesome conclusie





%TODO Er moet ook ergens nog komen dat we nested functies en arrays niet
%     hebben.


\end{document}
