\documentclass[a4paper]{article}
\usepackage[utf8]{inputenc}
\usepackage{listings}
\usepackage{graphicx}
\usepackage{amsmath}
\author{Rik van der Kooij <stdnmmr>\& \\
Richard Torenvliet - 6138861 \texttt{rich.riz@gmail.com}}
\title{Compiler Construction - Creating the CiViC Compiler}
\begin{document}
\abstract{Civicc Compiler}
\tableofcontents
\maketitle

\section{Introduction}
This document describes yet another implementation of the CiviC compiler for
the CiviC-VM.  The creation of the CiviC compiler consists of different phases
in order to parse the program text to assembly which in turn CiviC-VM can
interpret. Starting with creating the AST(Abstract Syntax Tree), which
describes the hierarchy of "Nodes". Next, creating the Lex and Yacc to parse a
file and assemble the AST, where phases like "desugaring", "type checking" and
"code generation".

\section{Abstract Syntax Tree}
The abstract syntaxtree is represented in XML form. In the XML
file(\texttt{src/global/ast.xml} the hierarchy and relation between nodes can be
expressed. The framework supports a certain syntax in order to describe the
tree. For example, an assign node(see reference document) is described by the following.

\begin{lstlisting}[language=XML]
 <node name="Assign">
  <sons>
    <son name="Let">
      <targets>
        <target mandatory="yes">
          <node name="Var" />
          <phases>
            <all />
          </phases>
        </target>
      </targets>
    </son>
    <son name="Expr">
      <targets>
        <target mandatory="yes">
          <set name="Expr" />
          <phases>
            <all />
          </phases>
        </target>
      </targets>
    </son>
  </sons>
  <attributes />
</node>
\end{lstlisting}

An important thing to note is that <sons> represent other nodes in the ast.xml,
and a relation between these nodes is created. To create a cyclic
relation, the node that is referred to can refer back to the parent node.

If a node is a ctype, those are described in the $\lt$attributes$\gt$ field.\\
\texttt{<type name="String" ctype="char*" init="NULL" copy="function" />}
\\
Also, the AST can be visually represented in a Directed Cyclic Graph, see
figure \ref{fig:ast.png}.

\begin{figure}[h!]
    \includegraphics[width=15cm]{../framework/doc/ast.png}
\label{fig:ast.png}
\end{figure}

The framework generates functions to create these nodes which can be
used in the Yaccer to construct the AST in memory. Function syntax of these
have the following format: \\
\texttt{TBmake<NODENAME>\_<SON\_NAME/ATTRIBUTE\_NAME>}

\section{Lex and Yacc}
In the Lex file we describe the patterns that we want to match as a token in
order to match a sequence of these to a certain node(exactly the nodes we
described in the ast.xml).
\begin{lstlisting}
#define FILTER(token) \
  global.col += yyleng;  \
  return( token);
"while"   { FILTER( WHILELOOP); }
\end{lstlisting}
A sequence of tokens in a certain order correspond to a node we defined in the
ast.xml.

\section{Desugaring}

\section{Type Checking}

\section{Code Generation}

\end{document}
